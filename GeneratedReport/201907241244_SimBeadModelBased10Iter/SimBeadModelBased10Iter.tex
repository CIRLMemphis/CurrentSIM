% acex13.tex/06/24/2013\documentclass[mathserif]{beamer}%

\documentclass[slidestop]{beamer}

\usepackage{amsxtra,amssymb,amsthm,amsmath,latexsym}

\usepackage{multicol}

\usepackage{beamerthemesplit}

%\usetheme{Darmstadt}

\numberwithin{equation}{section}

\newtheorem{thm}{Theorem}[]

\newcommand{\EN}{\mathcal{N}}

\newcommand{\na}{\nabla}

\newcommand{\R}{{\mathbb R}}

\newcommand{\ra}{\rightarrow}

\newcommand{\ue}{\infty}

\newcommand{\nc}{\newcommand}

\newcommand{\RRR}{\mathbb{R}^3}

\newcommand{\ol}{\overline}

\newcommand{\thmref}[1]{Theorem~\ref{#1}}

\newcommand{\lemref}[1]{Lemma~\ref{#1}}

\def\bee{\begin{equation*}}

\def\eee{\end{equation*}}

\def\be{\begin{equation}}

\def\ee{\end{equation}}

\nc{\Ba}{\Big(} \nc{\Bz}{\Big)} \nc{\pa}{\partial} \nc{\ti}{\times}

\nc{\n}{|} \nc{\al}{\alpha} \nc{\da}{\delta} \nc{\bs}{\backslash}

\nc{\ka}{\kappa} \nc{\Da}{\Delta} \nc{\si}{\sigma} \nc{\f}{\big(}

\nc{\g}{\big)}

%\nc{\n}{|}

\nc{\om}{\omega}

\begin{document}

%%%% Test - Einsatz  Kapitel 5 -

\part{ }

%XXXXXXXXXXXXXXXXXXXXXXXXXXXXXXXXXXXXXXXXXXXXXXXXXXXX

\title[CIRL Generated Report]{Generated report}

\author{}

\institute{}

\date{07-24-2019 12:51}

\frame{\titlepage}

\section{Report}

\frame[shrink=20]{ \frametitle{Simulation setup}
PSF type: PSFLutz
\begin{multicols}{2}
$distToCoverslip = -30\,\mu$\\ 
$ni = 1.518$\\ 
$ng = 1.518$\\ 
$tg = 170\,\mu$\\ 
$ns = 1.47$\\ 
$NA = 1.4$\\ 
$tiD = 190$\\ 
$niD = 1.518$\\ 
$ngD = 1.518$\\ 
$tgD = 170\,\mu$\\ 
$ts = -30\,\mu$\\ 
$lambda = 0.515\,\mu$
\end{multicols} 
\begin{multicols}{2}
$ X = 200, Y = 200, Z = 600$\\ 
$N_\phi = 3$\\ 
$N_\theta = 1$\\ 
$u_c = 5.4369$\\ 
$u_m = 0.19u_c$\\ 
$w_m = 0.078767$\\ 
$\varphi = (96.00, 13.00, -89.23)^{\circ}$\\ 
$\theta = (0.00,)^{\circ}$\\ 
$dXY = 0.10317\,\mu$\\ 
$dZ = 0.1\,\mu$\\ 
\end{multicols} 
Reconstruction method: GradientDescent
\begin{multicols}{2}
$numIt = 10.000000$\\ 
\end{multicols} 
} 
\frame{ \frametitle{Original object}
\begin{figure}[h!]
  \includegraphics[width=\linewidth]{OrigObject}
\end{figure}
}
\frame{ \frametitle{Simulated collected data at $\varphi = 96$}
\begin{figure}[h!]
  \includegraphics[width=\linewidth]{SimulatedData1}
\end{figure}
}
\frame{ \frametitle{Simulated collected data at $\varphi = 13$}
\begin{figure}[h!]
  \includegraphics[width=\linewidth]{SimulatedData2}
\end{figure}
}
\frame{ \frametitle{Simulated collected data at $\varphi = -89.231$}
\begin{figure}[h!]
  \includegraphics[width=\linewidth]{SimulatedData3}
\end{figure}
}
\frame{ \frametitle{Reconstructed object}
\begin{figure}[h!]
  \includegraphics[width=\linewidth]{ReconObject}
\end{figure}
}
\frame{ \frametitle{Original vs Reconstructed object, x axis}
\begin{figure}[h!]
  \includegraphics[width=\linewidth]{xObvsRecon}
\end{figure}
}
\frame{ \frametitle{Original vs Reconstructed object, z axis}
\begin{figure}[h!]
  \includegraphics[width=\linewidth]{zObvsRecon}
\end{figure}
}
\end{document}

